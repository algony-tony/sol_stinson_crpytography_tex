\documentclass[12pt,a4paper]{article}


\usepackage{amsmath}
\usepackage{mathrsfs}
\usepackage{bbding}
\usepackage{amsmath,amssymb,amsfonts,amsthm}
\usepackage{txfonts}
\usepackage{hyperref}
\usepackage{fancybox}
\hypersetup{colorlinks=true,linkcolor=blue}

\renewcommand{\baselinestretch}{1.2}
\renewcommand*{\qedsymbol}{\RectangleBold }

\newcommand{\pr}{\mathbf{Pr}}
\newcommand{\pp}{\mathscr{P}}
\newcommand{\bp}{\mathbf{P}}
\newcommand{\pc}{\mathscr{C}}
\newcommand{\bc}{\mathbf{C}}
\newcommand{\pk}{\mathscr{K}}
\newcommand{\bk}{\mathbf{K}}
\newcommand{\bs}{\mathbf{S}}
\newcommand{\lra}{\Longrightarrow}
\newcommand{\llra}{\Longleftrightarrow}

\newtheoremstyle{solution}% name
  {3pt}%      Space above
  {3pt}%      Space below
  {}%         Body font
  {}%         Indent amount (empty = no indent, \parindent = para indent)
  {\itshape}% Thm head font
  {:}%        Punctuation after thm head
  {.5em}%     Space after thm head: " " = normal interword space;
        %       \newline = linebreak
  {}%         Thm head spec (can be left empty, meaning `normal')

\theoremstyle{solution}
\newtheorem*{sol}{Solution}


\begin{document}

\title{Solutions of Cryptography Theory and Practice}
\author{Author}
\date{December 20, 2010}

%%  revision history
%\date{December 20, 2010}
%\date{Dec 20, 2010}
%\date{Dec 6, 2010}
%\date{Nov 22, 2010}
%\date{Nov 8, 2010}
%\date{Nov 1, 2010}
%\date{Oct 11, 2010}

\maketitle

\section{Classical Cryptography}

\textbf{1.21 \\[0.4ex] (c) Affine Cipher}

\begin{sol}
Since in the affine cryptosystem only $\phi(26) \times 26 =312$ cases for key $K=(a,b)$, list all the cases brutally with the help of computer and scan quickly, I found only one case seems meaningful:

\textup{ocanadaterredenosaieuxtonfrontestceintdefleuronsglorieuxcartonbrassaitporterlepeeils aitporterlacroixtonhistoireestuneepopeedesplusbrillan
tsexploitsettavaleurdefoitrempeeprotegeranosfoyersetnosdroits}

After breaking the long sentence, I found it should be the French version of Canada national anthem:
\begin{verse}
O Canada \\
Terre de nos a\"{i}eux\\
Ton front est ceint de fleurons glorieux!

Car ton bras sait porter l'\'{e}p\'{e}e  \\
Il sait porter la croix!  \\
Ton histoire est une \'{e}pop\'{e}e  \\
Des plus brillants exploits

Et ta valeur, de foi tremp\'{e}e \\
Prot\'{e}gera nos foyers et nos droits
\end{verse}
\end{sol}

\textbf{1.22 }
%Suppose that $p_1, \ldots, p_n$ and $q_1, \ldots, q_n$ are both probability distributions, and $p_1\geq \ldots \geq p_n$. Let $q'_1, \ldots, q'_n$ be any permutation of $q_1, \ldots, q_n$. Prove that the quantity
%$$\sum_{i=1}^n p_i q'_i$$
%is maximized when $q'_1\geq \ldots \geq q'_n$.

\begin{proof}
{\em (a)}
Obviously in the case $p_1\geq p_2\geq 0,\, q_1\geq q_2\geq 0$, the sum $p_1 q_1 +p_2 q_2$ is maximized.

Suppose $S=\sum_{i=1}^n p_i q'_i$ is maximized, and if there exists $i>j$ such that $ q'_i<q'_j$, permute the positions of the two numbers and, according to the statement of last paragraph, we get $S'={\sum\limits_{{0\leq h\leq n}\atop{h\neq i,j}} p_h q'_h}+p_i q'_j +p_j q'_i > S$, contradicts. So the only arrangement for $S$ is $q'_1\geq \ldots \geq q'_n$.
\end{proof}

%\textbf{1.27}
%
%\begin{proof}
%(a) From
%$$z_{m+i}=\sum_{j=0}^{m-1} c_j z_{i+j} \pmod2,$$
%we could get
%$$v_{m+i}=\sum_{j=0}^{m-1} c_j v_{i+j} \pmod2.$$
%\end{proof}

\section{Shannon's Theory}

\textbf{2.2}

\begin{proof}
$\pp= \pc = \pk= \{1,\ldots,n\}.\ \forall x \in \pp,\, \forall y \in \pc$
$$\pr [ Y=y ] = \sum_{k\in \pk } \pr [ K=k ] \pr [ x=d_k (y) ] = 1/n$$
so
\begin{align*}
\pr [x|y]&=\frac{\pr[x] \pr[y|x]}{\pr[y]}\\
          &=\frac{\pr[x] \frac{1}{n}}{\frac{1}{n}}\\
          &=\pr[x]
\end{align*}
that is this {\slshape Latin Square Cryptosystem} achieves perfect secrecy provided that every  key is used with equal probability.
\end{proof}

\textbf{2.3}

\begin{proof} Similarly as (2.2).\end{proof}

\textbf{2.5}

\begin{proof}
According to Theorem 2.4, this cryptosystem provides perfect secrecy iff every key is used with equal probability $1/|\pk|$, and for every $x\in \pp$ and $y\in \pc$, there is unique key $k\in \pk$, such that $y=e_k(x)$.

Due to perfect secrecy, we have
\begin{align*}
  &  &\pr[x|y]&=\pr[x]\\
  &\Longrightarrow &\frac{\pr[x] \pr[y|x]}{\pr[y]}&=\pr[x]\\
  &\Longrightarrow &  \pr[y]&=\pr[y|x]=\pr_{e_k(x)=y}[K=k]=1/|\pk|.\\
\end{align*}
that is every ciphertext is equally probable.
\end{proof}

\textbf{2.11}

\begin{proof}
(Perfect secrecy $\Rightarrow$ H($\bp|\bc$)=H($\bp$)).
\begin{align*}
H(\bp|\bc)&=-\sum_{y\in \bc}\sum_{x\in \bp} \pr[y]\pr[x|y]\log_2\pr[x|y]\\
          &=-\sum_{y\in \bc}\sum_{x\in \bp} \pr[y]\pr[x]\log_2\pr[x]\\
          &=H(\bp)\sum_{y\in \bc}\pr[y]\\
          &=H(\bp).
\end{align*}

($H(\bp|\bc)=H(\bp)\  \Rightarrow$ Perfect secrecy).

 According to Theorem 2.7,
\begin{gather*}
H(\bp,\ \bc)\leq H(\bp)+H(\bc)\\
\text{with equality iff $\bp$ and $\bc$ are independent random variables.}
\end{gather*}
so
\begin{align*}
   &H(\bp|\bc)=H(\bp)\\
\Longrightarrow\ \ &H(\bp,\ \bc)= H(\bp)+H(\bc)\\
\Longrightarrow\ \ &\bp \ and\  \bc \text{ are independent random variables}\\
\Longrightarrow\ \ &\forall x\in \bp,\ \forall y\in \bc,\ \pr[x|y]=\pr[x],
\end{align*}
that is the cryptosystem achieves perfect secrecy.
\end{proof}

\textbf{2.12}

\begin{proof}
We have
$$H(\bk,\,\bp,\,\bc)=H(\bp|\bk,\,\bc)+H(\bk,\,\bc)=H(\bk,\,\bc),$$
so
\begin{align*}
        &H(\bk,\,\bp,\,\bc)\geq H(\bp,\,\bc)\\
\lra\ \ &H(\bk,\,\bc)\geq H(\bp,\,\bc)\\
\lra\ \ &H(\bk|\bc)=H(\bk,\,\bc)-H(\bc)\geq H(\bp,\,\bc)-H(\bc)=H(\bp|\bc).
\end{align*}
\end{proof}

\textbf{2.13}

\begin{sol}
$H(\bp)=\frac{1}{2}+\frac{1}{3} \log_2 3+\frac{1}{6} \log_2 6 \approx 1.45915.$

$H(\bk)=\log_2 3 \approx 1.58496.$\medskip

\begin{center}
\begin{tabular}{ccc}
$\pr[1]$&=&2/9\\
$\pr[2]$&=&5/18\\
$\pr[3]$&=&1/3\\
$\pr[4]$&=&1/6\\
\end{tabular}
\end{center}\medskip


$H(\bc) \approx 1.95469.$

$H(\bk|\bc)=H(\bk)+H(\bp)-H(\bc)\approx 1.08942.$

\begin{center}
\begin{tabular}{rclp{4ex}rclp{4ex}rcl}
$\pr[K_1|1]$&=&3/4 && $\pr[K_2|1]$&=&0 && $\pr[K_3|1]$&=&1/4\\
$\pr[K_1|2]$&=&2/5 && $\pr[K_2|2]$&=&3/5 && $\pr[K_3|2]$&=&0\\
$\pr[K_1|3]$&=&1/6 && $\pr[K_2|3]$&=&1/3 && $\pr[K_3|3]$&=&1/2\\
$\pr[K_1|4]$&=&0 && $\pr[K_2|4]$&=&1/3 && $\pr[K_3|4]$&=&2/3\\
\end{tabular}
\end{center}

$H(\bp|\bc)\approx 1.08942.$
\end{sol}

\textbf{2.14}

\begin{sol}
$H(\bk)=\log_2 216,\ H(\bp)=\log_2 26,\ H(\bc)=\log_2 26,$
$$H(\bk|\bc)=H(\bk)+H(\bp)-H(\bc)=\log_2 216\approx 7.75489.$$
It is easy to evaluate:
$$H(\bk|\bp,\,\bc)=\log_2 12\approx 3.58496.$$
\end{sol}

\textbf{2.19}

\begin{proof}
A key in the product cipher $\bs_1 \times \bs_2$ has the form $(s_1,\,s_2)$, where $s_1,
\,s_2 \in \mathbb{Z}_{26}$,
$$e_{(s_1,\,s_2)}(x)=x+(s_1+s_2).$$
But this is precisely the definition of a key in the {\slshape Shift Cipher}. Further, the probability of a key in $\bs_1 \times \bs_2$ equals $\sum\limits_{i=0}^{25}\frac{1}{26}p_i=\frac{1}{26}\sum\limits_{i=0}^{25}=\frac{1}{26},$
which is also the probability of a key in the {\slshape Shift Cipher}. Thus $\bs_1 \times \bs_2$ is the {\slshape Shift Cipher}.
\end{proof}

\textbf{2.20}

\begin{proof}
{\em (a) }Similarly as (2.19).

{\em (b) }The number of keys in $\bs_3$ equals $26^{\mathrm{lcm}(m_1,\,m_2)}$. Form the following fact we can see that the keys in $\bs_1 \times \bs_2$ are lesser when $m_1 \nequiv 0 \pmod{m_2}$.

Notice
$$
\begin{array}{rl}
     & m_2 < m_1,~m_1 \nequiv 0 \pmod{m_2}\\
\lra & m_1+m_2 < 2 m_1 \leq \dfrac{m_2}{\gcd(m_1,~m_2)}m_2=l,~\text{where } l \triangleq \mathrm{lcm}(m_1,~m_2)
\end{array}
$$
the equation below, which essentially gives the representation of the keys of product cryptosystem $\bs_1 \times \bs_2$, would't always has a solution $(K_1,~K_2)=(x_1,\ldots,x_{m_1},y_1,\ldots,y_{m_2})\in \pk_1 \times \pk_2,$ for any selected key $K=(z_1,\ldots,z_l)\in \pk_3,$
$$
\begin{array}{ccc}
\begin{pmatrix}
\raisebox{-2ex}{\Large $I_{m_1}$} & I_{m_2}\\
        &I_{m_2}\\
\vdots & \vdots \\
         & I_{m_2}\\
\raisebox{2ex}{\Large $I_{m_1}$}& I_{m_2}
\end{pmatrix}~~
\begin{pmatrix}
x_1 \\ \vdots \\ x_{m_1} \\y_1 \\ \vdots \\y_{m_2}
\end{pmatrix}
&=
& \begin{pmatrix}  z_1\\ \vdots\\ \vdots\\ z_l  \end{pmatrix},
\end{array}
$$
because the coefficient matrix's column number is less than the row number. Conclusively, there must exist key $K\in \pk_3$ that not in $\pk_1 \times \pk_2.$
\end{proof}


\section{The RSA Cryptosystem and Factoring Integers}

\textbf{5.2}

\begin{proof}

{\em (a) }$r_i=q_{i+1}r_{i+1}+r_{i+2}\geq r_{i+1}+r_{i+2} \geq 2r_{i+2}.$

{\em (b) (c) }From (a), $m\leq \log_2 a +1 \sim \log a \sim \log b.$
\end{proof}

\textbf{5.3}

\begin{sol}

 {\em (a)} $17^{-1} \equiv 6\ \ \pmod{101}.$

{\em (b)} $357^{-1} \equiv 1075\ \ \pmod{1234}.$

{\em (c)} $3125^{-1} \equiv 1844\ \ \pmod{9987}.$
\end{sol}

\textbf{5.4}

\begin{sol} $8 \times 93 -13\times 57=3.$  \end{sol}

\textbf{5.5}

\begin{sol}
It's easy to see that
$$\chi^{-1}(x,\,y,\,z)=70x+21y+15z\ \  \pmod{105}.$$
So
$$\chi^{-1}(2,~2,~3)=17.$$
\end{sol}

\textbf{5.6}

\begin{sol}
Similarly we have
$$\chi^{-1}(x,\,y,\,z)=(13\times 26 \times 27)x+(25^2\times 27)y+(14\times25\times26)z\ \ \pmod{25\times26\times27}.$$
So
$$\chi^{-1}(12,~9,~23)= 470687.$$
\end{sol}

\textbf{5.7}

\begin{sol}
Similarly we have
$$\chi^{-1}(x,~y)=(50\times 101)x+(51\times 99)y\ \ \pmod{99\times 101}.$$
$13^{-1} \equiv 61\ \ \pmod{99},~15^{-1}\equiv 27\ \ \pmod{101}.$ So
$$
\begin{array}{rl}
 &\left\{
   \begin{array}{rclc}
   13x&\equiv &4  &\pmod{99}\\
   15x&\equiv &56 &\pmod{101}
   \end{array} \right. \\
   &\\
\Longleftrightarrow &
\left\{
   \begin{array}{rclc}
   x&\equiv &46  &\pmod{99}\\
   x&\equiv &98 &\pmod{101}
   \end{array}\right.\\
   &\\
\lra &x=7471.
\end{array}
$$
\end{sol}

\textbf{5.9}

\begin{proof}
$$
\begin{array}{cl}
    & \alpha\nequiv \pm 1 \pmod p,~\alpha~\text{is~primitive~element~modulo }p\\
\Longleftrightarrow &
\text{the order of }\alpha \in \mathbb{Z}_p^* \text{ is } 2q.\\
\llra &
\alpha\nequiv \pm 1 \pmod p,~\alpha^q \equiv -1 \pmod p.\\
    &\text{(Because the order of $\alpha$ could only be 2, $q$ or $2q$.)}
\end{array}
$$
\end{proof}

\textbf{5.11}

\begin{proof}
By showing $x^{\lambda(n)} =1 \pmod n,~\forall x\in \mathbb{Z}_n,$ we can prove the encryption and decryption are still inverse operations in this modified cryptosystem.

Since we have
$$\mathbb{Z}_n \cong \mathbb{Z}_p \oplus \mathbb{Z}_q,$$
and the isomorphism is $\Psi (h)=(h\pmod{p},h\pmod{q}),$ it only needs to show$x^{\lambda(n)} =1 \pmod p,$ and $x^{\lambda(n)} =1 \pmod q.$
$$
\begin{array}{rcl}
x^{\lambda(n)}& \equiv &{(x^{p-1})}^{\frac{q-1}{\gcd (p-1,q-1)}}\pmod p\\
&\equiv &(1)^{\frac{q-1}{\gcd (p-1,q-1)}}\pmod p\\
&\equiv &1 \pmod p.
\end{array}
$$
For $q$ is the same. So complete the proof.
\end{proof}

\begin{sol}{\em (b)  }$p=37,~q=79,~b=7.$ \end{sol}

{\slshape Modified cryptosystem.} $\lambda (n)=468,~a=67.$

{\slshape Original cryptosystem.} $\phi (n)=2808,~a=2407.$

\textbf{5.13}

\begin{proof}

{\em (a) }Again use the isomorphism
$$
\mathbb{Z}_n \cong \mathbb{Z}_p \oplus \mathbb{Z}_q,
$$
and note that $x^{p-1}\equiv 1 \pmod{p}$ and $x^{q-1}\equiv 1 \pmod{q}$ for all $x \neq 0,$
we get
$$
\begin{array}{cclcc}
\mathbb{Z}_n & \rightarrow & \multicolumn{1}{c}{\mathbb{Z}_p \oplus \mathbb{Z}_q}  & \rightarrow & \mathbb{Z}_n\\
y^d \pmod{n}         & \rightarrow  & ~~~(y^d \pmod{p},~y^d \pmod{q})& \rightarrow & M_p q x_p+M_q p x_q \pmod{n}\\
&&=(y^{d_p} \pmod{p},~y^{d_q} \pmod{q})&&\\
&&=(x_p,x_q)&&
\end{array}
$$
so complete the proof.

{\em (b) }$p=1511,~q=2003,$ and $d=1234577$.

After some calculation, we get $d_p=907,~d_q=1345,~M_p=777,$ and $M_q=973.$

{\em (c) }$y=152702,~y^{d_p}=242 \pmod{1511},~y^{d_q}=1087 \pmod{2003},~y^d=1443247 \pmod{n=3026533}.$
\end{proof}

\textbf{5.14}

\begin{sol}
If $\gcd(y,n)\neq 1,$ we get a factor $p=\gcd(y,n)$ of $n$, and another factor $q=\dfrac{n}{\gcd(y,n)}$, consequently, $d_K$ and $x=d_K(y)$ can be calculated explicitly.

Now suppose $\gcd(y,n)=1$, computer $\hat{y}=y^{-1}\pmod{n}$, which is chosen as the attack allows to know the plaintext $\hat{x}$, it is easy to see that $x=\hat{x}^{-1}\pmod{n}$.
\end{sol}

\textbf{5.18}

\begin{proof}
The isomorphism
$$
\mathbb{Z}_n \cong \mathbb{Z}_p \oplus \mathbb{Z}_q
$$
is used again to show
$$
e_K(x)=x^a=x \llra (x^a\mod p,x^a\mod q)=(x\mod p,x\mod q).
$$
And
\begin{center}
$\mathbb{F}_p^*$ is multiplicative cyclic group of order $p-1,$\\
$\mathbb{F}_q^*$ is multiplicative cyclic group of order $q-1,$\\
\end{center}
deduce that
$$
\#S_p=\#\{x\in \mathbb{F}_p^* | x^a =x\}=\gcd(a-1,p-1),
$$
$$
\#S_q=\#\{x\in \mathbb{F}_q^* | x^a =x\}=\gcd(a-1,q-1),
$$
so
$$
\begin{array}{rcl}
\#S&=&\# \{x\in \mathbb{Z}_n|x\neq0,(x^a\mod p,x^a\mod q)=(x\mod p,x\mod q)\}\\
&=&\#(S_p \times S_q)\\
&=&\gcd(a-1,p-1) \times \gcd(a-1,q-1)\\
&=&\gcd(b-1,p-1) \times \gcd(b-1,q-1),
\end{array}
$$
the last equation can be easily seen from the equality
$$
ab \equiv 1 \pmod{(p-1)(q-1)},\text{ i.e. }(a-1)b \equiv -(b-1) \pmod{(p-1)(q-1)}.
$$
\end{proof}


\textbf{5.22}

\begin{proof}
{\em (a) }$G(n)$ is a subgroup of $\mathbb{Z}_n^*$, because $1\in G(n)$, and $\forall x,y \in G(n),$
$$
\left( \frac{x^{-1}}{n}\right) \equiv \left( \frac{x}{n}\right)^{-1} \equiv a^{(-1)(n-1)/2} \pmod{n},
$$
$$
\left( \frac{xy}{n}\right) \equiv \left( \frac{x}{n}\right) \left( \frac{y}{n}\right) \equiv x^{(n-1)/2} y^{(n-1)/2} \equiv (xy)^{(n-1)/2} \pmod{n},
$$
i.e. $x^{-1}\in G(n),~xy\in G(n)$.

If $G(n)\neq \mathbb{Z}_n^*$, obviously we have
$$
|G(n)| \leq \frac{|\mathbb{Z}_n^*|}{2}\leq \frac{n-1}{2}.
$$

{\em (b) }On one hand,
$$
\left( \frac{a}{n}\right) \equiv \left( \frac{1+p^{(k-1)}q}{p^k}\right) \left( \frac{1+p^{(k-1)}q}{q}\right) \equiv 1 \pmod{n},
$$
on the other hand,
$$
a^{(n-1)/2} \equiv (1+p^{(k-1)}q)^{(n-1)/2} \equiv 1+\frac{(p^kq-1)p^{(k-1)}q}{2} \equiv 1+\frac{p^{(k-1)}q}{2} \nequiv 1 \pmod{n},
$$
so
$$
\left( \frac{a}{n}\right) \nequiv a^{(n-1)/2} \pmod{n}.
$$

{\em (c) }On one hand, we have
$$
\left( \frac{a}{n}\right) \equiv \left( \frac{a}{p_1}\right) \left( \frac{a}{p_2 p_3 \ldots p_s}\right) \equiv (-1)\times 1\equiv 1 \pmod{n},
$$
on the other hand, from the definition of $a$, we have
$$
a^{(n-1)/2} \equiv 1 \pmod{p_2 p_3 \ldots p_s}
$$
so $a^{(n-1)/2} \nequiv -1 \pmod{n}$, because otherwise we can deduce that $a^{(n-1)/2} \equiv -1 \pmod{p_2 p_3 \ldots p_s}$, contradicts.

{\em (d) }If $n$ is odd and composite, suppose its factorization is $n=p_1^{k_1}p_2^{k_2}\ldots p_s^{k_s}$, then there must exit $n_1$ of the form $p^k q$ or the form $p_1 \ldots p_s$ for some primes $p,q,p_1, \ldots, p_s$, such that $n=n_1 n_2$. Now if $G(n)=\mathbb{Z}_n^*$, i.e. $\forall a \in \mathbb{Z}_n^*,\left( \frac{a}{n}\right) \equiv a^{(n-1)/2} \pmod{n}$, so we have $\forall a \in \mathbb{Z}_n^*,\left( \frac{a}{n}\right) \equiv a^{(n-1)/2} \pmod{n_1}$, which contradicts the results of (b) and (c). Thus we must have $G(n)\neq \mathbb{Z}_n^*$, and, resulting from (a), $G(n)\leq (n-1)/2$.

{\em (e) } Summarize the above: we finally prove that the error probability of the Solovay-Strassen primality test is at most 1/2.
\end{proof}


\textbf{5.23}

\begin{proof}
{\em (b)} The average number of trials to achieves success is
$$
\sum_{n=1}^{\infty} (n\times p_n),
$$
now apply the following identity
\begin{align*}
\frac{1}{(1-x)^2} &=\frac{\mathrm{d}}{\mathrm{d}x}\left(\frac{1}{1-x}\right) \\
&=\frac{\mathrm{d}}{\mathrm{d}x}\left( \sum_{n=0}^{\infty}x^n \right) \\
&=\sum_{n=1}^{\infty}nx^{n-1} \hspace{5ex} \text{ for } |x|<1,
\end{align*}
we have
\begin{align*}
\sum_{n=1}^{\infty} (n\times p_n)&=\sum_{n=1}^{\infty} (n\times \epsilon^{n-1}(1-\epsilon))\\
&=(1-\epsilon) \sum_{n=1}^{\infty}n{\epsilon}^{n-1}\\
&=\frac{1}{1-\epsilon}.
\end{align*}

{\em (c) }Suppose $n$ is the number of iteraitons required in order to reduce the possibility of failure to at most $\delta$, so we have
\begin{align*}
{\epsilon}^n & \leq \delta,\\
\intertext{i.e.}
n & \leq \log_{\epsilon}{\delta}\\
 & \leq \left\lceil \frac{\log_2 {\delta}}{\log_2 \epsilon} \right\rceil
\end{align*}
\end{proof}

\textbf{5.24}

\begin{proof}
{\em (a) }

\medskip
\fbox{
\begin{minipage}{12cm}
\textbf{Algorithm: }Lifting($a,b,n$)\\[0.5ex]
$b_0 ~\leftarrow~ b ~\mod p$\\
$b_1 ~\leftarrow~ b_0^{-1} ~\mod p$\\
$h ~\leftarrow~ b^2-a ~\mod p^{i-1}$\\
$x ~\leftarrow~ -h\frac{p+1}{2}b_1 ~\mod p$\\
\textbf{return(}$x$\textbf{)}
\end{minipage}
}

\medskip
The reason for the algorithm is following: suppose the lift of $b$ to $\mathbb{Z}_{p^i}$ is $b+xp^{i-1}$, then
\begin{align*}
&&(b+xp^{i-1})^2 &\equiv a \pmod{p^i}\\
\lra && b^2+2bxp^{i-1} &\equiv a \pmod{p^i}\\
\lra &&2bxp^{i-1} &\equiv -hp^{i-1} \pmod{p^i}\\
\lra &&x &\equiv -h2^{-1}b^{-1} \pmod{p^i}\\
\lra &&x &\equiv -h\frac{p+1}{2}b_1 \mod {p}
\end{align*}

{\em (b)}
\begin{align*}
6^2 &\equiv 17 \pmod{19}\\
(6+11\cdot 19)^2=215^2 &\equiv 17 \pmod{19^2}\\
(6+11 \cdot 19+2 \cdot 19^2)=937^2&\equiv 17 \pmod{19^3}
\end{align*}

{\em (c) }This result can be directly deduced from Hensel's Lemma.
\end{proof}

\textbf{5.29}

\begin{proof}
{\em (a) }
\begin{align*}
n+d^2 &=pq+d^2\\
 &=p(p+2d)+d^2\\
 &=(p+d)^2\\
\end{align*}

{\em (b) }Let $h^2=n+d^2$, then $(h+d)(h-d)=h^2-d^2=n$.

{\em (c) }
$$n=2189284635403183,$$
$$d=9,$$
$$h=46789792,$$
$$h^2=2189284635403264=n+81,$$
so
$$(h+d)(h-d)=46789783\cdot 46789801=2189284635403183=n.$$
\end{proof}

\textbf{5.30}

\begin{sol}
$$n=36581,~a=14039,~b=4679,$$
$$ab-1=65688481=2^5\cdot 2052765$$
$$r=2052765.$$
When $w=9983,~\gcd(w,n)=1,$
$$w^r\equiv 35039 \mod {n}$$
$$w^{2r}\equiv 36580\equiv-1 \mod {n}.\textbf{~~~~~Fail.}$$
When $w=13461,~\gcd(w,n)=1,$
$$w^r\equiv 11747 \mod {n}$$
$$w^{2r}\equiv 8477 \mod {n}$$
$$w^{4r}\equiv 14445 \mod {n}$$
$$w^{8r}\equiv 1 \mod {n}$$
$$\gcd(14444,n)=157,~\gcd(14446,n)=233,$$
$$233 \cdot 157=36581=n$$
\end{sol}

\textbf{5.32}

\begin{sol}
$$n=317940011,~b=77537081,$$
the continued fraction expansion of $b/n$ is
$$[0, 4, 9, 1, 19, 1, 1, 15, 3, 2, 3, 71, 3, 2].$$
The partial convergents of this continued fraction are as follows:
$$
\begin{array}{rcl}
~[0,4]  &=& 1/4\\
~[0, 4, 9] &=&9/37\\
~[0, 4, 9, 1] &=& 10/41\\
\end{array}
$$
The $n'$ and equations are listed:
$$
\begin{array}{rll}
1/4 & n'=310148323 & X^2 - 7791689X + 317940011 = 0,\mbox{ No integer solutions.}\\
9/37 & n'=2868871996/9& \\
10/41 & n'=317902032& X^2 - 37980X + 317940011 = 0,~X_1=12457,~X_2=25523
\end{array}
$$
Finally we have the factors of $n=12457\cdot25523.$
\end{sol}


\textbf{5.33}

\begin{sol}
{\em (a)} $n=41989,~y=e_K(32767)=16027.$

{\em (b)} The four possible decryption of this ciphertext $y$ is
$$
\begin{array}{cc}
x_1=32767, &x_2=7865, \\
x_3=18837, &x_4=21795.
\end{array}
$$
%{mathematica command: Solve[y (y + B) - 16027 == 0 && Modulus == n, y, Mode -> Modular]}%
\end{sol}

\section{Public-key Cryptography and Discrete Logarithms}

\textbf{6.4}
\begin{sol}
{\em (a)} Suppose $\alpha$ is a primitive element modulo $p$, since $\mathrm{ord}(\alpha)|p(p-1)$, to show $\alpha$ is a primitive element modulo $p^2$ is equivalent to show $\alpha^{p-1} \nequiv 1 \pmod{p^2},$ and $\alpha^p \nequiv 1 \pmod{p^2}$. And we already have
$$
\alpha \text{ is a primitive element modulo } p \lra \alpha^p \nequiv 1 \pmod{p^2}.
$$
If $\alpha^{(p-1)}\equiv(\alpha+p)^{(p-1)}\equiv 1 \pmod{p^2},$ then we can get $p|\alpha,$ contradicted with the choice of $\alpha$, so at least one of $\alpha$ or $\alpha +p$ is a primitive element modulo $p^2.$

{\em(b)} We only need to check there is no such non-trivial factor $h$ of $p-1$ that $3^h\equiv 1 \pmod{29}$ to confirm 3 is a primitive element modulo 29, and to check $3^{28}\nequiv 1\pmod{29^2}$ to confirm 3 is a primitive element modulo $29^2$.

{\em(c)} 14.

{\em(d)} $24388=2^2\cdot7\cdot 13\cdot67$

$\log_3 3344 =18762 $ in $\mathbb{Z}_{24389}^*.$
%{Mathematica Command: MultiplicativeOrder[3, 24389, {3344}]}
\end{sol}

\textbf{6.6}

\begin{sol}
{\em (a)} $\alpha=2,~p=227,$
$$
\begin{array}{rcccl}
\alpha^{32}&\equiv&176 \pmod{227}&\equiv&2^4 \cdot11 \pmod{227}\\
\alpha^{40}&\equiv&110 \pmod{227}&\equiv&2\cdot5\cdot11 \pmod{227}\\
\alpha^{59}&\equiv&60 \pmod{227}&\equiv&2^2\cdot3\cdot5 \pmod{227}\\
\alpha^{156}&\equiv&28 \pmod{227}&\equiv&2^2\cdot7 \pmod{227}
\end{array}
$$

{\em (b)}
$$
\begin{array}{ll}
\log3=46,&\log5=11,\\
\log7=154,&\log11=28.
\end{array}
$$

{\em (c)}
$$173\cdot 2^{177}\equiv 168 \pmod{227}\equiv 2^3\cdot3\cdot7,$$
$$\log173=26.$$
\end{sol}

\textbf{6.7}

\begin{proof}
{\em (a)} From the following isomorphism
$$\mathbb{Z}_n \cong \mathbb{Z}_p \oplus \mathbb{Z}_q,$$
directly we have
$$\mathrm{ord}_n(\alpha)=\mathrm{lcm}(\mathrm{ord}_p(\alpha),\mathrm{ord}_q(\alpha)).$$

{\em (b)} Suppose $\alpha_p,~\alpha_q$ be generators of $\mathbb{Z}_p,~\mathbb{Z}_q$ respectively. Chinese Remainder Theorem assures the existence of the follow equations
$$
\left\{
\begin{array}{lll}
x&\equiv&\alpha_p \pmod{n}\\
x&\equiv&\alpha_q \pmod{n}\\
\end{array}
\right. ,
$$
denote the solution by $\alpha,$ now applying (a),
$$\mathrm{ord}_n(\alpha)=\mathrm{lcm}(\mathrm{ord}_p(\alpha),\mathrm{ord}_q(\alpha))=\frac{\phi(n)}{d}.$$

{\em (c)}
\begin{align*}
 &~\alpha^a \equiv \alpha^n \pmod{n}\\
\lra~&~\alpha^{n-a}\equiv 1 \pmod{n}\\
\lra~&~\exists k\in \mathbb{N}^*,~\text{s.t. } k\frac{\phi(n)}{2}=(n-a),
\end{align*}
from $k\frac{\phi(n)}{2}\leq n,~p>3,~q>3,$ we get $k \leq 2$, and from $0\leq a\leq \phi(n)/2-1$, we get $k=2,$ i.e. $n-a=\phi(n).$

{\em (d)} The roots of the equation
$$x^2+(n-\phi(n)+1)x+n=0$$
are $p,~q$.
\end{proof}

\textbf{6.10}

\begin{sol}$x^5+x^3+1$ is irreducible,
\begin{align*}
x^5+x^4+1&=\left(1+x+x^2\right) \left(1+x+x^3\right),\\
x^5+x^4+x^2+1&=(1+x) \left(1+x+x^4\right).
\end{align*}
%{Mathematica Command: Factor[x^5 + x^4 + x^2 + 1, Modulus -> 2]}
\end{sol}

\textbf{6.11}

\begin{sol}
{\em (a)} $(x^4+x^2)(x^3+x^2+1)\equiv 1+x+x^2+x^3+x^4$, in $\mathbb{F}_2[x]/(x^5+x^2+1).$

{\em (b)} $(x^3+x^2)^{-1}\equiv 1+x+x^2$, in $\mathbb{F}_2[x]/(x^5+x^2+1).$

{\em (c)} $x^{25}\equiv 1+x^3+x^4,$ in $\mathbb{F}_2[x]/(x^5+x^2+1).$
\end{sol}

\textbf{6.12}

\begin{sol}
The plaintext is ``GaloisField".
\end{sol}

\textbf{6.13}

\begin{sol}
(a) $\#E(\mathbb{F}_{71})=72$.

%(1, 32)(1, 39)(2, 31)(2, 40)
%(3, 22)(3, 49)(4, 5)(4, 66)(5, 4)
%(5, 67)(6, 26)(6, 45)(12, 8)(12, 63)
%(13, 26)(13, 45)(15, 9)(15, 62)(19, 27)
%(19, 44)(20, 5)(20, 66)(21, 3)(21, 68)
%(22, 30)(22, 41)(23, 19)(23, 52)(25, 22)
%(25, 49)(27, 0)(31, 32)(31, 39)(33, 1)
%(33, 70)(34, 23)(34, 48)(35, 14)(35, 57)
%(36, 12)(36, 59)(37, 33)(37, 38)(39, 32)
%(39, 39)(41, 7)(41, 64)(43, 22)(43, 49)
%(47, 5)(47, 66)(48, 11)(48, 60)(49, 24)
%(49, 47)(52, 26)(52, 45)(53, 0)(58, 27)
%(58, 44)(61, 15)(61, 56)(62, 0)(63, 17)
%(63, 54)(65, 27)(65, 44)(66, 18)(66, 53)
%(69, 35)(69, 36)
%Total number of points is 72 

(b) According to Theorem 6.1, $E(\mathbb{F}_{71})$ could only be isomorphism to either $\mathbb{Z}_{72}$ or $\mathbb{Z}_{36} \times \mathbb{Z}_2$. Now from the fact that $E(\mathbb{F}_{71})$ has four points of order 2---they are $(27, 0), (53, 0), (62, 0)$ and the special point $\mathcal{O}$, we can conclude that
$$E(\mathbb{F}_{71}) \cong \mathbb{Z}_{36} \times \mathbb{Z}_2.$$

(c) From the above discussion we know the maximum order of a point on E is 36, and the point $(3, 22)$ has this order.
\end{sol}

\textbf{6.14}

\begin{proof}
Let the three district roots of $x^3+ax+b\equiv 0 \pmod{p}$ be $x_1,~x_2,~x_3$. Hence points $(x_1,~0),~(x_2,~0),~(x_3,~0)$ and the special point $\mathcal{O}$ are the only four points of order 2 on $E$, and according to the group theory, they must be isomorphism to $\mathbb{Z}_2 \times \mathbb{Z}_2$ as groups. Therefore, group $E(\mathbb{F}_p)$ containing the above four points as a subgroup can not be cyclic.
 \end{proof}

\textbf{6.15}

\begin{sol}
(a) Directly from the calculation.

(b) Obviously.

(c) $(1,~20),~(1,~53),~(2,~21),~(2,~52),~(71,~17),~(71,~56),~(72,~29),~(72,~44)$ and the special point $\mathcal{O}$.
%{Mathematica Command: Solve[3 x^4 + 6*34* x^2 - (34)^2 == 0 && Modulus == 73, x, Mode -> Modular]}%
\end{sol}

\textbf{6.17}

\begin{sol}
(a) $Q=mP=(8,~15).$

(b)

\begin{tabular}{lllll}
Ciphertext   & ((18, 1), 21) & ((3, 1), 18) & ((17, 0), 19) & ((28, 0), 18)\\
Step 1 ($kP$)& ((18, 17), 21) & ((3, 3), 18) & ((17, 26), 19) & ((28, 6), 18)\\
Step 2 ($kQ$)& (15, 8)  & (2, 9) & (30, 29) & (14, 19)\\
Plaintext    &  20 & 9 & 12 & 5\\
\end{tabular}

(c) TILE.
\end{sol}

\textbf{6.18}

\begin{sol}
(a) $87=(-1)\cdot 2^0+(-1)\cdot 2^3+(-1)\cdot 2^5+2^7$.

(b) $87P=(102,~88)$.
\end{sol}

\textbf{6.20}

\begin{sol}
$log_5 896=147$ in $\mathbb{Z}_{1103}$
\end{sol}


\textbf{6.21}

\begin{proof}
(a) Since $a^{(p-1)/2} \equiv 1 \pmod{p}$, as square root $a^{(p-1)/4} \equiv \pm 1 \pmod{p}$.

(b)
\begin{align*}
 a & \equiv 1 \cdot a \pmod{p}\\
   & \equiv a^{(p-1)/4} \cdot a \pmod{p}\\
   & \equiv a^{(p+3)/4} \pmod{p}\\
   & \equiv (a^{(p+3)/8})^2 \pmod{p}.\\
\end{align*}

(c) Since $p \equiv 5 \pmod{8}$, we have $\left( \frac{2}{p}\right) =-1$, i.e. $2^{(p-1)/2} \equiv -1 \pmod{p}$.
\begin{align*}
 a & \equiv (-1)\cdot (-1) \cdot a \pmod{p}\\
   & \equiv 2^{(p-1)/2} \cdot a^{(p-1)/4} \cdot a \pmod{p}\\
   & \equiv 2^{-2} \cdot 4^{(p+3)/4} \cdot a^{(p+3)/4} \pmod{p}\\
   & \equiv (2^{-1}(4a)^{(p+3)/8})^2 \pmod{p}.\\
\end{align*}

(d)Based on above discussions, it's known that it is possible to computer square root modulo $p$. Furthermore, since $p\equiv 5 \pmod{8}$, we have $\left( \frac{-1}{p}\right) =1$, that is $L_1(\beta)=L_1(-\beta)$ for all $\beta \in \mathbb{Z}_p ^*$. So $L_2(\beta)=L_1(\pm \sqrt{\beta/\alpha^{L_1(\beta)}})$ can be computered efficiently.
\end{proof}

\section{Signature Schemes}

\textbf{7.1}
$p=31847,~\alpha=5,~\beta=25703,~x_1=8990,~(\gamma_1,~\delta_1)=(23972,~31396),~x_2=31415,~(\gamma_2,~\delta_2)=(23972,~20481).$

\begin{sol}
Notice that $\gamma_1=\gamma_2$, and further calculation shows
$$(\delta_1-\delta_2)^{-1} \equiv 22317 \mod (p-1),$$
thus
\begin{align*}
a\gamma_1 &= x_2-\delta_2(x_1-x_2)/(\delta_1-\delta_2) \mod(p-1) \\
&\equiv 23704 \pmod{(p-1)},\\
\intertext{$a$ has two solutions to this equation, but only one satisfies the eqution $\alpha^a\equiv \beta \pmod{p}$, that is}
a&\equiv7459\mod(p-1), \\
%a&\equiv 23382\mod(p-1), \text{wrong}\\
\intertext{for $k$, we have}
k&=(x_2-a\gamma_1)\delta_2 \mod(p-1) \\
&\equiv  1165 \pmod{(p-1)},\\
\end{align*}
\end{sol}

\textbf{7.4}

\begin{sol}
(a) All the notations in this exercise are in accordance with those in Section 7.3.
\begin{align*}
\beta^{\lambda}\lambda^\mu & \equiv \beta^\lambda (\gamma^h \alpha^i \beta^j)^{\delta\lambda(h\gamma-j\delta)^{-1}} \pmod{p}\\
    &\equiv (\beta^{(h\gamma-j\delta)} \gamma^{h\delta} \alpha^{i\delta}\beta^{j\delta})^{(h\gamma-j\delta)^{-1}\lambda} \pmod{p}\\
    &\equiv (\beta^{h\gamma}\gamma^{h\delta}\alpha^{i\delta})^{\lambda(h\gamma-j\delta)^{-1}} \pmod{p}\\
    &\equiv (\alpha^{xh}\alpha^{i\delta})^{\lambda(h\gamma-j\delta)^{-1}} \pmod{p}\\
    &\equiv \alpha^{x'} \pmod{p}.
\end{align*}

(b)
\begin{align*}
\lambda&=363,\\
\mu &=401,\\
x' &=385.
\end{align*}
\end{sol}

\textbf{7.5}

\begin{sol}
(a) $\delta=0$ means $x\equiv a\gamma \pmod{(p-1)}$, and since $x,~\gamma$ are public, it's easy to computer $a$.

(b) $\gamma=0$ means $k\delta \equiv x\pmod{q}$, $k$ is thus determined. Using this $k$ any forgery can be made.

(c) In the ECDSA, $r=0$ makes  $k$ is available, $s=0$ yields the secret key $m$.
\end{sol}


\textbf{7.6}

\begin{sol}
(a)
$$\verb|ver|_K (x,~(\gamma,~\delta))=\text{true} \Leftrightarrow \beta^\delta \gamma^\gamma \equiv \alpha^x \pmod{p}.$$

(b) One inversion operation in finite field is saved in each procedure of signing.

\end{sol}

\textbf{7.10}

\begin{proof}
(a) We only need to show
$$(\alpha ^{e_1} \beta ^{e_2} \mod p) \mod q =\gamma,$$
where
\begin{align*}
e_1& =x \delta^{-1} \mod q \\
e_2& =\gamma \delta ^{-1} \mod q.
\end{align*}
\begin{align*}
(\alpha ^{e_1} \beta ^{e_2} \mod p) \mod q & =(\alpha ^{x \delta^{-1}} \beta ^{\gamma \delta^{-1}} \mod p) \mod q\\
   & =(\alpha ^{\lambda^{-1}} \beta ^{\gamma x^{-1}\lambda^{-1}\mod q} \mod p) \mod q\\
     & =(\alpha \beta^{\gamma x^{-1}})^{\lambda^{-1}\mod q}\mod p \mod q\\
     & =((\alpha \epsilon^{\gamma})^{\lambda^{-1} \mod q})\mod p \mod q\\
     & =\gamma
\end{align*}
\end{proof}

\textbf{7.13}

\begin{sol}
(a) The public key $B=mA=(24,~44)$.

(b)$q=131,~x=10,~k=75,~kA=(88,~55),~r=88,~s=60.$

(c)$w=107,~i=22,~j=115,~iA=(91,~18),~jB=(18,~62),~iA+jB=(88,~55),~u=88=r.$

\end{sol}

\textbf{7.14}

\begin{proof}
Obviously.
\end{proof}

\textbf{7.15}
$p=467,~\alpha=4,~\beta=449,~a=101,~y=25,~x=157,~e_1=46,~e_2=123,~f_1=198,~f_2=11.$

\begin{sol}

Bob's challenges:
\begin{align*}
c&=280 \\
C&=17. \\
\end{align*}
Alice's responses:
\begin{align*}
d&=193 \\
D&=21. \\
\end{align*}
Verification:
\begin{align*}
x^{e_1} \alpha^{e_2}&\equiv 22 \nequiv d \pmod{p}\\
x^{f_1} \alpha^{f_2}&\equiv 276 \nequiv D \pmod{p}\\
(d\alpha^{-e_2})^{f_1} \equiv 137 \equiv & ~(D\alpha^{-f_2})^{e_1} \pmod{p}.\text{~~~~Forgery.}\\
\end{align*}
\end{sol}

\textbf{7.17}

\begin{sol}
(a)
\begin{equation*}
\left\{ \begin{aligned}
          a_1&=1118-42b_1 \pmod{p}\\
          a_2&=1449-42b_2 \pmod{p}\\
          \end{aligned} \right.
\end{equation*}

(b)
\begin{equation*}
\left\{ \begin{aligned}
          a_1+42b_1&=1118 \pmod{p}\\
          a_2+42b_2&=1449 \pmod{p}\\
          a_1+969b_1&=899 \pmod{p}\\
          a_2+969b_2&=471 \pmod{p}\\
          \end{aligned} \right.
\end{equation*}
\begin{equation*}
\lra \left\{ \begin{aligned}
          a_1&=724 \pmod{p}\\
          a_2&=3109 \pmod{p}\\
          b_1&=2816 \pmod{p}\\
          b_2&=2602 \pmod{p}.\\
          \end{aligned} \right.
\end{equation*}
% {Mathematica command: Solve[{a1 + 42 b1 == 1118, a2 + 42 b2 == 1449, a1 + 969 b1 == 899,a2 + 969 b2 == 471 && Modulus == 3467},{a1, b1, a2, b2},Mode -> Modular]}
\end{sol}

\textbf{7.18}

\begin{sol}
(a)
$$\gamma_1\gamma_2 ^x \equiv 1943 \pmod{p},$$
$$\alpha^{y_1}\beta^{y_2} \equiv 1943 \pmod{p},$$
so,this signature is valid.

(b)
\begin{align*}
y_1&=917, \\
y_2&=1983, \\
a_0&=2187.\\
\end{align*}
\end{sol}


\end{document}















